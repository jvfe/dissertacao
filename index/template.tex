% ------------------------------------------------------------------------
% ------------------------------------------------------------------------
% Modelo UFSC para Trabalhos Academicos (tese de doutorado, dissertação de
% mestrado) utilizando a classe abntex2
%
% Autor: Alisson Lopes Furlani
% 	Modificações:
%	- 27/08/2019: Alisson L. Furlani, add pacote 'glossaries' para listas
% - 30/10/2019: Alisson L. Furlani, adjusted some spacing errors and changed math fonts
% - 17/01/2019: Alisson L. Furlani, updated certification page
% - 03/03/2020: Luiz F. P. Droubi, change file to be used as a template with R.
% ------------------------------------------------------------------------
% ------------------------------------------------------------------------

\documentclass[
	% -- opções da classe memoir --
	12pt,				% tamanho da fonte
	%openright,			% capítulos começam em pág ímpar (insere página vazia caso preciso)
	oneside,			% para impressão no anverso. Oposto a twoside
	a4paper,			% tamanho do papel.
	% -- opções da classe abntex2 --
	chapter=TITLE,		% títulos de capítulos convertidos em letras maiúsculas
	section=TITLE,		% títulos de seções convertidos em letras maiúsculas
	%subsection=TITLE,	% títulos de subseções convertidos em letras maiúsculas
	%subsubsection=TITLE,% títulos de subsubseções convertidos em letras maiúsculas
	% -- opções do pacote babel --
	english,			% idioma adicional para hifenização
	%french,				% idioma adicional para hifenização
	%spanish,			% idioma adicional para hifenização
	brazil				% o último idioma é o principal do documento
	]{abntex2}

\usepackage{setup/ufrnthesisA4-alf}

$for(bibliography)$
\addbibresource{$bibliography$}
$endfor$

\usepackage[table]{xcolor}
\let\newfloat\undefined
\usepackage{floatrow}
\floatsetup[table]{capposition=top}
\floatsetup[figure]{capposition=top}

\newcommand{\pkg}[1]{{\normalfont\fontseries{b}\selectfont #1}}
\let\proglang=\textsf
\let\code=\texttt

$if(highlighting-macros)$
$highlighting-macros$
$endif$

\newcommand{\bcenter}{\begin{center}}
\newcommand{\ecenter}{\end{center}}

\newcommand{\bapendices}{\begin{apendicesenv}}
\newcommand{\eapendices}{\end{apendicesenv}}

\newcommand{\banexos}{\begin{anexosenv}}
\newcommand{\eanexos}{\end{anexosenv}}

% ---
% Filtering and Mapping Bibliographies
% ---
\DeclareSourcemap{
	\maps[datatype=bibtex]{
		% remove fields that are always useless
		\map{
			\step[fieldset=abstract, null]
			\step[fieldset=pagetotal, null]
		}
		% remove URLs for types that are primarily printed
%		\map{
%			\pernottype{software}
%			\pernottype{online}
%			\pernottype{report}
%			\pernottype{techreport}
%			\pernottype{standard}
%			\pernottype{manual}
%			\pernottype{misc}
%			\step[fieldset=url, null]
%			\step[fieldset=urldate, null]
%		}
		\map{
			\pertype{inproceedings}
			% remove mostly redundant conference information
			\step[fieldset=venue, null]
			\step[fieldset=eventdate, null]
			\step[fieldset=eventtitle, null]
			% do not show ISBN for proceedings
			\step[fieldset=isbn, null]
			% Citavi bug
			\step[fieldset=volume, null]
		}
	}
}
% ---

% ---
% Informações de dados para CAPA e FOLHA DE ROSTO
% ---
% FIXME Substituir 'Nome completo do autor' pelo seu nome.
\autor{$author$}
% FIXME Substituir 'Título do trabalho' pelo título da trabalho.
\titulo{$title$}
% FIXME Substituir 'Subtítulo (se houver)' pelo subtítulo da trabalho.
% Caso não tenha substítulo, comente a linha a seguir.
$if(subtitle)$
  \subtitulo{$subtitle$}
$endif$
% FIXME Substituir 'XXXXXX' pelo nome do seu
% orientador.
\orientador{$advisor$}
% FIXME Se for orientado por uma mulher, comente a linha acima e descomente a linha a seguir.
% \orientador[Orientadora]{Nome da orientadora, Dra.}
% FIXME Substituir 'XXXXXX' pelo nome do seu
% coorientador. Caso não tenha coorientador, comente a linha a seguir.
$if(altadvisor)$
\coorientador{$altadvisor$}
$endif$
% FIXME Se for coorientado por uma mulher, comente a linha acima e descomente a linha a seguir.
% \coorientador[Coorientadora]{XXXXXX, Dra.}
% FIXME Substituir '[ano]' pelo ano (ano) em que seu trabalho foi defendido.
\ano{$date_year$}
% FIXME Substituir '[dia] de [mês] de [ano]' pela data em que ocorreu sua defesa.
\data{$date_day$ de $date_month$ de $date_year$}
% FIXME Substituir 'Local' pela cidade em que ocorreu sua defesa.
\local{$local$}
\instituicaosigla{$inst_short$}
\instituicao{$institution$}
% FIXME Substituir 'Dissertação/Tese' pelo tipo de trabalho (Tese, Dissertação).
\tipotrabalho{$doc_type$}
% FIXME Substituir '[mestre/doutor] em XXXXXX' pela grau adequado.
\formacao{$degree$}
% FIXME Substituir '[mestrado/doutorado]' pelo nivel adequado.
\nivel{$level$}
% FIXME Substituir 'Programa de Pós-Graduação em XXXXXX' pela curso adequado.
\programa{$department$}
% FIXME Substituir 'Campus XXXXXX ou Centro de XXXXXX' pelo campus ou centro adequado.
\centro{$division$}
\preambulo
{%
\imprimirtipotrabalho~apresentada~ao~\imprimirprograma~da~\imprimirinstituicao.
}
% ---

% ---
% Configurações de aparência do PDF final
% ---
% alterando o aspecto da cor azul
\definecolor{blue}{RGB}{41,5,195}
% informações do PDF
\makeatletter
\hypersetup{
     	%pagebackref=true,
		pdftitle={\@title},
		pdfauthor={\@author},
    	pdfsubject={\imprimirpreambulo},
	    pdfcreator={LaTeX with abnTeX2},
		pdfkeywords={ufsc, latex, abntex2},
		colorlinks=true,       		% false: boxed links; true: colored links
    	linkcolor=black,%blue,          	% color of internal links
    	citecolor=black,%blue,        		% color of links to bibliography
    	filecolor=black,%magenta,      		% color of file links
		urlcolor=black,%blue,
		bookmarksdepth=4
}
\makeatother
% ---

% ---
% compila a lista de abreviaturas e siglas e a lista de símbolos
% ---

% Declaração das siglas
\siglalista{MS}{metagenômica \textit{shotgun}}
\siglalista{16S}{sequenciamento da subunidade ribossomal 16S de genomas bacterianos}
\siglalista{TDM}{transtorno depressivo maior}
\siglalista{DNA}{Ácido Desoxirribonucléico}
\siglalista{OTU}{unidades taxonômicas operacionais - do inglês \textit{operational taxonomic units}}
\siglalista{ASV}{variantes de sequência amplicon - do inglês \textit{amplicon sequence variant}}


% Declaração dos simbolos
\simbololista{C}{\ensuremath{C}}{Circunferência de um círculo}
\simbololista{pi}{\ensuremath{\pi}}{Número pi}
\simbololista{r}{\ensuremath{r}}{Raio de um círculo}
\simbololista{A}{\ensuremath{A}}{Área de um círculo}


% compila a lista de abreviaturas e siglas e a lista de símbolos
\makenoidxglossaries

% ---

% ---
% compila o indice
% ---
\makeindex
% ---

% ----
% Início do documento
% ----
\begin{document}

% Seleciona o idioma do documento (conforme pacotes do babel)
%\selectlanguage{english}
\selectlanguage{brazil}

% Retira espaço extra obsoleto entre as frases.
\frenchspacing

% Espaçamento 1.5 entre linhas
\OnehalfSpacing

% Corrige justificação
%\sloppy

% ----------------------------------------------------------
% ELEMENTOS PRÉ-TEXTUAIS
% ----------------------------------------------------------
% \pretextual %a macro \pretextual é acionado automaticamente no início de \begin{document}
% ---
% Capa, folha de rosto, ficha bibliografica, errata, folha de apróvação
% Dedicatória, agradecimentos, epígrafe, resumos, listas
% ---
% ---
% Capa
% ---
\imprimircapa
% ---

% ---
% Folha de rosto
% (o * indica que haverá a ficha bibliográfica)
% ---
\imprimirfolhaderosto*
% ---

% ---
% Inserir a ficha bibliografica
% ---
% http://ficha.bu.ufsc.br/
% \begin{fichacatalografica}
% 	\includepdf{Ficha_Catalografica.pdf}
% \end{fichacatalografica}
% ---

% ---
% Inserir folha de aprovação
% ---
\begin{folhadeaprovacao}
	\OnehalfSpacing
	% \centering
	\begin{center}
	\imprimirautor\\%
	\vspace*{10pt}
	\textbf{\imprimirtitulo}%
	\ifnotempty{\imprimirsubtitulo}{:~\imprimirsubtitulo}\\%
	%		\vspace*{31.5pt}%3\baselineskip
	\vspace*{\baselineskip}
	\end{center}
	%\begin{minipage}{\textwidth}
	\imprimirtipotrabalho~apresentada~ao~\imprimirprograma~da~\imprimirinstituicao\\
	%\end{minipage}%
	\bigskip\newline
	\textbf{Área de Concentração}:~Bioinformática\newline%
	\textbf{Linha de Pesquisa}:~Biologia de Sistemas\newline%
	\imprimirorientadorRotulo:~\imprimirorientador\newline%
	\begin{flushright}
	Natal, \imprimirdata.
	\end{flushright}
	\vspace*{\baselineskip}
  % $for(examiner)$
  % $examiner.title$ $examiner.name$, $examiner.degree$\\
  % $examiner.institution$ - $examiner.inst_short$\\
  % \vspace*{\baselineskip}
  % $endfor$

	\vspace*{2\baselineskip}
	% \begin{minipage}{\textwidth}
	% 	Certificamos que esta é a \textbf{versão original e final} do trabalho de conclusão que foi julgado adequado para obtenção do título de \imprimirformacao.\\
	% \end{minipage}
	%    \vspace{-0.7cm}
	\centering{\textbf{BANCA EXAMINADORA}}
	\assinatura{\OnehalfSpacing Prof. Dr. \imprimirorientador \\ \footnotesize{\imprimirinstituicao \\ (Presidente)}}
  $for(examiner)$
  \assinatura{$examiner.title$ $examiner.degree$ $examiner.name$ \\ \footnotesize{$examiner.institution$ \\ (Examinador $examiner.role$)}}
  $endfor$
	%	\ifnotempty{\imprimircoorientador}{
	%	\assinatura{\imprimircoorientador \\ \imprimircoorientadorRotulo \\
	%		\imprimirinstituicao~--~\imprimirinstituicaosigla}
	%	}
	% \newpage
	\vspace*{\fill}
	\centering
\end{folhadeaprovacao}
% ---

% ---
% Dedicatória
% ---
% \begin{dedicatoria}
% 	\vspace*{\fill}
% 	\noindent
% 	\begin{adjustwidth*}{}{5.5cm}
% 		\raggedleft
% 		$dedication$
% 	\end{adjustwidth*}
% \end{dedicatoria}
% ---

% ---
% Agradecimentos
% ---
\begin{agradecimentos}
	$thanks$
\end{agradecimentos}
% ---

% ---
% Epígrafe
% ---
\begin{epigrafe}
	\vspace*{\fill}
	\begin{flushright}
		\textit{$epigrafe$}
	\end{flushright}
\end{epigrafe}
% ---

% ---
% RESUMOS
% ---

% resumo em português
\setlength{\absparsep}{18pt} % ajusta o espaçamento dos parágrafos do resumo
\begin{resumo}
	\SingleSpacing
  $abstract$

  \textbf{Palavras-chave}:
  $for(palavras-chave)$
  $palavras-chave$.
  $endfor$
\end{resumo}

% resumo em inglês
\begin{resumo}[Abstract]
	\SingleSpacing
	\begin{otherlanguage*}{english}
		$foreignabstract$

		\textbf{Keywords}:
	  $for(keyword)$
    $keyword$.
    $endfor$
	\end{otherlanguage*}
\end{resumo}

%% resumo em francês
%\begin{resumo}[Résumé]
% \begin{otherlanguage*}{french}
%    Il s'agit d'un résumé en français.
%
%   \textbf{Mots-clés}: latex. abntex. publication de textes.
% \end{otherlanguage*}
%\end{resumo}
%
%% resumo em espanhol
%\begin{resumo}[Resumen]
% \begin{otherlanguage*}{spanish}
%   Este es el resumen en español.
%
%   \textbf{Palabras clave}: latex. abntex. publicación de textos.
% \end{otherlanguage*}
%\end{resumo}
%% ---

{%hidelinks
	\hypersetup{hidelinks}
	% ---
	% inserir lista de ilustrações
	% ---
	\pdfbookmark[0]{\listfigurename}{lof}
	\listoffigures*
	\cleardoublepage
	% ---

	% ---
	% inserir lista de quadros
	% ---
	% \pdfbookmark[0]{\listofquadrosname}{loq}
	% \listofquadros*
	% \cleardoublepage
	% ---

	% ---
	% inserir lista de tabelas
	% ---
	\pdfbookmark[0]{\listtablename}{lot}
	\listoftables*
	\cleardoublepage
	% ---

	% ---
	% inserir lista de abreviaturas e siglas (devem ser declarados no preambulo)
	% ---
	\imprimirlistadesiglas
	% ---

	% ---
	% inserir lista de símbolos (devem ser declarados no preambulo)
	% ---
	\imprimirlistadesimbolos
	% ---

	% ---
	% inserir o sumario
	% ---
	\pdfbookmark[0]{\contentsname}{toc}
	\tableofcontents*
	\cleardoublepage

}%hidelinks
% ---

% ---

% ----------------------------------------------------------
% ELEMENTOS TEXTUAIS
% ----------------------------------------------------------
\textual

$body$

% ----------------------------------------------------------
% Glossário
% ----------------------------------------------------------
%
% Consulte o manual da classe abntex2 para orientações sobre o glossário.
%
%\glossary

%---------------------------------------------------------------------
% INDICE REMISSIVO
%---------------------------------------------------------------------
%\phantompart
%\printindex
%---------------------------------------------------------------------

\end{document}
